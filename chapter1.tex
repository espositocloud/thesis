\chapter{Introduction}\label{introduction}

When the Internet and the Web started to spread all over the world, a first era of cloud computing\cite{A Break in the Clouds: Towards a Cloud Definition} emerged in order to offer services like mail exchange and search engines.  In the last decade,  the exponential increment of connections, connected devices and people, led to the need of services for collaborative work and data synchronization between more devices, introducing new challenges.

Today cloud computing technologies are driving the evolution of Information Technology in numerous ways, leading to innovative approaches of managing software development and deployment processes, but also of the application architectures their own.

Since usually the evolution proceeds in gradual steps, most of current solutions are generally too bounded to past paradigms, and do not fit current necessities in an clear and elegant way, making the management a complicated task and reducing the scaling potential of applications.  Other solutions are proprietary, therefore they are not publicly available and do not contribute, at least directly, to global progress.

In early 2013, with the announce of Docker\footnote{https://www.docker.com/docker-engine}, cloud computing saw a big step forward thanks to containers as the atomic unit of cloud platforms.  Containers provide an isolated environment for running applications, or a part of them, and add a lightweight layer that decouples the application itself from the underling system.  In June 2014, Google, with around 10 years of experience in managing containers in production, embraced Docker as emerging industrial standard, releasing Kubernetes\footnote{http://kubernetes.io/}, a container orchestration, as core tool for managing whole applications at scale.

The aim of this work is to propose a radical step forward, reusing the growing ecosystem of free software built around containers, in order to provide a platform\cite{Developing Software Online With Platform-as-a-Service Technology} (PaaS) on top of an existing cloud infrastructure\cite{A survey and taxonomy of infrastructure as a service and web hosting cloud providers} (IaaS), with the goal of provide a suitable environment for running applications\cite{Understanding service-oriented software} (SaaS) in a modern and scalable way.  The proposed solution aims to provide a lightweight approach for a small and modern company, using Docker and Kubernetes, including projects like CoreOS\footnote{https://coreos.com/} by the namesake company, RedHat's OpenShift\footnote{http://www.openshift.org/} and HashiCorp's Terraform\footnote{https://terraform.io/}, but also InfluxDB\footnote{https://influxdb.com/}, Heka\footnote{http://hekad.readthedocs.org/} and Grafana\footnote{http://grafana.org/} as monitoring stack, and other minor projects.  All these tools are quite young since they are around 1~3 years old.

\textit{Gasista Felice}\footnote{https://befair.github.io/gasistafelice/} has been used as case of study, so it has been containerized for running on top of our platform, providing scalability analysis and some benchmarks.

This project proved that a clear and modern solution could be used for deploying common applications in a small environment, providing all the necessary for management, scaling and guaranteeing the governance through monitoring.  Finally, the project has been developed as multiple sub-projects publicly released as free software on the GitHub\footnote{https://github.com/} portal.

\section{Organization}\label{organization}

Chapter 2 introduces the working environment in which this thesis was born and the relative technological context.

Chapter 3 introduces some container core concepts, then exemplified their application in a specific use case.

Chapter 4 describes how to bootstrap an IaaS/PaaS in order to provide a production environment.

Chapter 5 describes the stack of PaaS showing how applications could take advantage of it.

Chapter 6 shows how to control, monitor, analyze, and visualize data
about applications.

Chapter 7 tries to draw conclusions about the entire work, and presents some potential further developments of this project.