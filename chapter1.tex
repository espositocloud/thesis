\chapter{Introduction}\label{introduction}

Today there is the \emph{cloud}, where some kind of services are
delegated by third-party providers. There are structured mainly 3
different kind of services, depends on abstraction level:

\begin{itemize}
\itemsep1pt\parskip0pt\parsep0pt
\item
  \emph{Infrastructure as a Service} (IaaS) for raw compute, network,
  storage and other low-level needs
\item
  \emph{Platform as a Service} (PaaS) for running applications by
  software developers
\item
  \emph{Software as a Service} (SaaS) for provide applications to end
  users
\end{itemize}

Anyway, even if cloud resoles a necessity sometimes there is the need of
avoid \emph{outsourcing}, or using directly cloud technologies for
providing services to customers.

\section{Goal}\label{goal}

Luckily, today there is a vast availability of free software for
reproduce some cloud services.

Starting from an overview of software development, systems management,
software architectures and application environments, it will be provided
some guidelines and researched existing available solutions, with the
goal of following the process of PaaS management built on top of an
IaaS, in a modern and scalable way, guaranteeing the governance through
monitoring.

The final goal is providing SaaS through a suitable environment for
running applications. For this purpose, it will be used an application
as case of study: \emph{Gasista Felice}.

This solution will integrate:

\begin{itemize}
\itemsep1pt\parskip0pt\parsep0pt
\item
  configuration of a IaaS
\item
  describing in a declarative way the PaaS resources
\item
  monitoring and benchmarking the live management of the whole platform
  when the application will be in execution
\end{itemize}

The examined and studied solution covers the current state of art, in
particular consists of \emph{Kubernetes} developed from Google's
Engineers, RedHat's \emph{OpenShift}, but also \emph{CoreOS} from the
namesake company, and HashiCorp's \emph{Terraform}.

All the project comes as free software and are released on the GitHub
portal.

\section{Organization}\label{organization}

Chapter 2 introduces the working environment in which this thesis was
born and the technological context for which was conceived.

Chapter 3 introduces containers' core concepts and the application to
the use case.

Chapter 4 describes how to bootstrap a IaaS/PaaS providing a production
environment.

Chapter 5 describes the stack of PaaS showing how applications could
take advantage on it.

Chapter 6 shows how to control, monitor, analyze, and visualize data
about applications.

Chapter 7 does a summary of this experience, and present some potential
evolution of this project in order to achieve further targets.