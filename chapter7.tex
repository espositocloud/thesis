\chapter{Conclusions}\label{conclusions}

Containers and emerging ecosystem around them are questioning the whole
IT industry about the role of the OS today. There are several new
opportunities for radically enhance the development and deployment
workflow, software architecture and application environments.

Docker is a new and quite immature runtime, but the future seems very prosperous. Kubernetes on the other side, has been adopted as scheduling and orchestration
framework by other PaaS beyond OpenShift, like Deis,
and from lower-level components, like Mesos\footnote{https://mesos.apache.org/}.

\section{Future Developments}\label{future-developments}

At this point, there are a large variety of potential improvements, mainly about \textit{high availability}\cite{HighAvailability}.

Applications running on top of the platform that use traditional
relational database such as PostgreSQL, could be configured for high
availability also for data\footnote{https://www.compose.io/articles/high-availability-for-postgresql-batteries-not-included/}.

From the monitoring and logging point of view, an
alert system could be feasible by an Heka alert encoder plug-in, or via a
dedicated software like Sentry\footnote{https://getsentry.com/} or Bosun\footnote{https://bosun.org/}.

From the cluster perspective, there is the possibility of rolling update
or auto-scaling of the cluster itself. In fact over time,
the stack components should be updated, and the applications deployed on cluster
could need globally more resources. A way to do this is monitoring the
global load of the cluster, and instruct Terraform for varying the number
of nodes accordingly to it.

Beyond the single cluster, there is an emerging proposal for
\textit{cluster federation}, called Ubernetes\footnote{http://kubernetes.io/v1.0/docs/proposals/federation.html}, that enables
high-availability with geographically distributed Kubernetes clusters.

With that global distributed infrastructure, it's possible to build
geographically distributed applications on top of databases like
Consul\footnote{https://consul.io/} for key-value storage or CockroachDB\footnote{http://www.cockroachlabs.com/} for more complex cases.