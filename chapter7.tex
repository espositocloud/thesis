\chapter{Conclusions}\label{conclusions}

\section{Summary}\label{summary}

Containers and emerging ecosystem around them are questioning the whole
IT industry about the role of the OS today. There are several new
opportunity for radically enhance the development and operations
workflow, software architecture and application environments.

Docker is a new and pretty immature runtime, but the promises around it
there are. Kubernetes on the other side, as scheduling and orchestration
framework, has been adopted from other PaaS beyond OpenShift, like Deis,
and from lower-level components, like Mesos.

\section{Future Developments}\label{future-developments}

At this point, there are a large variety of potential improvements.

From the monitoring and logging point of view, could be enabled an
alerting system directly via Heka alert encoder plug-in, or via a
dedicated software like \emph{Sentry} or \emph{Bosun}.

The main point is \emph{high availability}\cite{HighAvailability}.

Applications running on top of the platform that use traditional
relational database such as PostgreSQL, could be configured for high
availability also for data
(https://www.compose.io/articles/high-availability-for-postgresql-batteries-not-included/).

From the cluster perspective, there is the possibility of rolling update
or auto-scaling of the cluster itself. In fact with passing the time,
the stack should be updated, and the applications deployed on cluster
could need globally more resources. A way to do this is monitoring the
global load of the cluster, and instruct Terraform for vary the number
of nodes accordingly to it.

Beyond the single cluster, there is an emerging proposal for
\emph{cluster federation}, called Ubernetes, that enables
high-availability with geographically distributed Kubernetes clusters.

With that global distributed infrastructure, it's possible build
geographically distributed applications built on top of databases like
Consul for key-value storage or CockroachDB for more complex cases.